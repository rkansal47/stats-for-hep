\documentclass[12pt]{report}

\usepackage{setspace}

\usepackage{amsthm}

\ifdefined\HCode%
    \usepackage{amssymb}
\else
    % palatino font (doesn't work with tex4ht, but it can be specified in the CSS):
    \usepackage{newpxtext}
    \usepackage{newpxmath}
\fi

\usepackage[sort&compress,numbers]{natbib}
% \setcitestyle{maxcitenames=6}

\usepackage{array}
\usepackage{tabularx}
\usepackage{booktabs}
\usepackage{caption}
\usepackage{ragged2e}
\usepackage{microtype}
\usepackage{graphicx}
\usepackage{xcolor}
\usepackage{xparse}
\usepackage{pifont}
\usepackage{nicefrac}
\usepackage{adjustbox}
\usepackage{xspace}
% \usepackage{resizegather}
\usepackage[margin=1.2in]{geometry} % Adjust the margin size here

\usepackage{enumerate}
\usepackage[shortlabels]{enumitem}
\setlist[enumerate]{nosep}


\usepackage{amsmath}
% \usepackage{amssymb}
\usepackage{amsfonts}
\usepackage{slashed}
\usepackage{cancel}
\usepackage{dsfont}
\usepackage{braket}
\usepackage{multirow}
\usepackage{needspace}
\usepackage[bottom]{footmisc}
\usepackage{changepage}
\usepackage{titlesec}
\usepackage{rotating}
% \usepackage{tocloft}
\usepackage[normalem]{ulem}
\usepackage{tikz}
\usepackage[compat=1.1.0]{tikz-feynman}
\usepackage{simplewick}

\definecolor{awesome-emerald}{HTML}{00A388}
\definecolor{awesome-skyblue}{HTML}{0395DE}
\definecolor{awesome-red}{HTML}{DC3522}
\definecolor{awesome-pink}{HTML}{EF4089}
\definecolor{awesome-orange}{HTML}{FF6138}
\definecolor{awesome-nephritis}{HTML}{27AE60}
\definecolor{awesome-concrete}{HTML}{95A5A6}
\definecolor{awesome-darknight}{HTML}{131A28}
\colorlet{awesomelinks}{awesome-skyblue}

\theoremstyle{definition}
\newtheorem{definition}{Definition}[chapter]

\theoremstyle{definition}
\newtheorem{example}{Example}[chapter]

\theoremstyle{definition}
\newtheorem{theorem}{Theorem}[chapter]

\numberwithin{equation}{chapter}
% \numberwithin{figure}{section}
% \numberwithin{table}{section}

\usepackage{hyperref}
\hypersetup{colorlinks=true,
            linkcolor=awesomelinks,
            urlcolor=awesomelinks,
            citecolor=awesomelinks,}

\usepackage{doi}

\sloppy
\bibliographystyle{apsrev4-2}

% Required information
\title{Notes on Statistics for HEP}
\author{Raghav Kansal}

% Start the document
\begin{document}

\newcommand{\diag}{\ensuremath{\mathrm{diag}}\xspace}
\newcommand{\identity}{\ensuremath{\mathds{1}}\xspace}
\newcommand{\mustequal}{\ensuremath{\overset{!}{=}}\xspace}
\newcommand{\tr}[1]{\ensuremath{\mathrm{Tr}\left[#1\right]}\xspace}
\newcommand{\cnicefrac}[2]{%
  \ifdefined\HCode  % nicefrac not well supported by tex4ht
    \frac{#1}{#2}
  \else
    \nicefrac{#1}{#2}
  \fi
}
\newcommand{\cslashed}[1]{%
  \ifdefined\HCode  % slashed not supported by tex4ht
    \cancel{#1}
  \else
    \slashed{#1}
  \fi
}
\newcommand{\cvec}[1]{\mathbf{#1}}

% chi'^2
\newcommand{\chipt}{\chi^{\prime 2}}

\NewDocumentCommand{\SO}{ o }{%
  \ensuremath{\mathrm{SO}\IfValueT{#1}{(#1)}}\xspace
}
\NewDocumentCommand{\so}{ o }{%
  \ensuremath{\mathfrak{so}\IfValueT{#1}{(#1)}}\xspace
}
\NewDocumentCommand{\SU}{ o }{%
  \ensuremath{\mathrm{SU}\IfValueT{#1}{(#1)}}\xspace
}
\NewDocumentCommand{\su}{ o }{%
  \ensuremath{\mathfrak{su}\IfValueT{#1}{(#1)}}\xspace
}
\NewDocumentCommand{\UU}{ o }{%
  \ensuremath{\mathrm{U}\IfValueT{#1}{(#1)}}\xspace
}
\NewDocumentCommand{\OO}{ o }{%
  \ensuremath{\mathrm{O}\IfValueT{#1}{(#1)}}\xspace
}
\NewDocumentCommand{\EE}{ o }{%
  \ensuremath{\mathrm{E}\IfValueT{#1}{(#1)}}\xspace
}

\newcommand{\suu}{\ensuremath{\mathrm{SU}(2)\times\mathrm{U}(1)}\xspace}

\newcommand{\svecp}[1]{\ensuremath{#1_{\cvec{p}}\,}}
\newcommand{\ophat}[1]{\ensuremath{\hat #1_{\cvec{p}}\,}}
\newcommand{\ophatd}[1]{\ensuremath{\hat #1^\dagger_{\cvec{p}}\,}}

\newcommand{\abs}[1]{\left\lvert #1 \right\rvert}
\newcommand{\mus}{\ensuremath{\mu\text{s}}\xspace}
\newcommand{\mum}{\ensuremath{\mu\text{m}}\xspace}
\newcommand{\dd}{\ensuremath{\mathrm{d}}\xspace}

\newcommand{\cl}{\ensuremath{c_{\lambda}}\xspace}
\newcommand{\cvv}{\ensuremath{c_{2V}}\xspace}
\newcommand{\pt}{\ensuremath{p_{\mathrm{T}}}\xspace}
\newcommand{\Pl}{\textrm{l}\xspace}
\newcommand{\Pq}{\textrm{q}\xspace}
\newcommand{\Pg}{\textrm{g}\xspace}
\newcommand{\Pc}{\textrm{c}\xspace}
\newcommand{\Pb}{\textrm{b}\xspace}
\newcommand{\PH}{\textrm{H}\xspace}
\newcommand{\PV}{\textrm{V}\xspace}
\newcommand{\PW}{\textrm{W}\xspace}
\newcommand{\PZ}{\textrm{Z}\xspace}
\newcommand{\PQt}{\textrm{t}\xspace}
\newcommand{\PGe}{\textrm{e}\xspace}
\newcommand{\PGn}{\ensuremath{\nu}\xspace}
\newcommand{\PGm}{\ensuremath{\mu}\xspace}
\newcommand{\PGt}{\ensuremath{\tau}\xspace}
\newcommand{\PGg}{\ensuremath{\gamma}\xspace}
\newcommand{\PQq}{\textrm{q}\xspace}
\newcommand{\PQAq}{\bar{\PQq}\xspace}
\newcommand{\PQc}{\textrm{c}\xspace}
\newcommand{\PQAc}{\bar{\PQc}\xspace}
\newcommand{\PQb}{\textrm{b}\xspace}
\newcommand{\PQAb}{\bar{\PQb}\xspace}
\newcommand{\PHH}{\textrm{HH}\xspace}
\newcommand{\qqbar}{\ensuremath{\PQq\PQAq}\xspace}
\newcommand{\ccbar}{\ensuremath{\PQc\PQAc}\xspace}
\newcommand{\bbbar}{\ensuremath{\PQb\PQAb}\xspace}
\newcommand{\CL}{\textrm{CL}\xspace}
\newcommand{\CLs}{\ensuremath{\text{CL}_{\text{s}}}\xspace}

\newcommand{\PX}{\textrm{X}\xspace}
\newcommand{\PY}{\textrm{Y}\xspace}
\newcommand{\PG}{\textrm{G}\xspace}
\newcommand{\PA}{\textrm{A}\xspace}
\newcommand{\Pa}{\textrm{a}\xspace}
\newcommand{\Ph}{\textrm{h}\xspace}
\newcommand{\PHsm}{\ensuremath{\PH_{125}\xspace}}

\newcommand{\kapt}{\ensuremath{\kappa_{\PQt}}\xspace}
\newcommand{\kapl}{\ensuremath{\kappa_{\lambda}}\xspace}
\newcommand{\kapv}{\ensuremath{\kappa_{\PV}}\xspace}
\newcommand{\kapvv}{\ensuremath{\kappa_{2\PV}}\xspace}
\newcommand{\mH}{\ensuremath{m_{\PH}}\xspace}
\newcommand{\mHH}{\ensuremath{m_{\PH\PH}}\xspace}
\newcommand{\sigmaHH}{\ensuremath{\sigma_{\PH\PH}\xspace }}
\newcommand{\BR}{\ensuremath{\mathcal{B}(\HH\to\bbbar\ww)}}
\newcommand{\HH}{\ensuremath{{\PH\PH}}\xspace}
\newcommand{\HHH}{\ensuremath{{\PH\PH\PH}}\xspace}
\newcommand{\HVV}{\ensuremath{{\PH\PV\PV}}\xspace}
\newcommand{\HHVV}{\ensuremath{{\PH\PH\PV\PV}}\xspace}
\newcommand{\HHY}{\ensuremath{{\PH\PH/\PY}}\xspace}
\newcommand{\VH}{\ensuremath{{\PV\PH}}\xspace}
\newcommand{\ggH}{\ensuremath{{\Pg\Pg\PH}}\xspace}
\newcommand{\ttH}{\ensuremath{{\ttbar\PH}}\xspace}
\newcommand{\ggHH}{\ensuremath{{\Pg\Pg\PH\PH}}\xspace}
\newcommand{\qqHH}{\ensuremath{{\Pq\Pq\PH\PH}}\xspace}
\newcommand{\bbbb}{\ensuremath{\bbbar\bbbar}\xspace}
\newcommand{\mSD}{\ensuremath{m_{\text{SD}}}\xspace}
\newcommand{\ssbar}{\PQs{}\PAQs\xspace}
\newcommand{\alpS}{\ensuremath{\alpha_s}\xspace}

\newcommand{\TODO}[1]{\textcolor{red}{TODO: #1}}
\newcommand{\gbb}{\ensuremath{\Pg\to\bbbar}\xspace}
\newcommand{\hbb}{\ensuremath{\PH\to\bbbar}\xspace}
\newcommand{\hqq}{\ensuremath{\PH\to\qqbar}\xspace}
\newcommand{\htata}{\ensuremath{\PH\to\PGtp\PGtm}\xspace}
\newcommand{\qqqq}{\ensuremath{4\Pq}\xspace}
\newcommand{\ww}{\ensuremath{\PW\PW}\xspace}
\newcommand{\zz}{\ensuremath{\PZ\PZ}\xspace}
\newcommand{\VV}{\ensuremath{\PV\PV}\xspace}
\newcommand{\bbtautau}{\ensuremath{\bbbar\PGt\PGt}\xspace}
\newcommand{\bbgg}{\ensuremath{\bbbar\PGg\PGg}\xspace}
\newcommand{\bbww}{\ensuremath{\bbbar\PW\PW}\xspace}
\newcommand{\bbvv}{\ensuremath{\bbbar\PV\PV}\xspace}
\newcommand{\bbvvq}{\ensuremath{{\bbbar(\VV\to\qqqq)}}\xspace}
\newcommand{\wwq}{\ensuremath{\PW\PW\to\qqqq}\xspace}
\newcommand{\vqq}{\ensuremath{\PV\to\PQq\PQq}\xspace}
\newcommand{\VVq}{\ensuremath{\PV\PV\to\qqqq}\xspace}
\newcommand{\hvv}{\ensuremath{\PH\to\VV}\xspace}
\newcommand{\yvv}{\ensuremath{\PY\to\VV}\xspace}
\newcommand{\yww}{\ensuremath{\PY\to\ww}\xspace}
\newcommand{\hyvv}{\ensuremath{\PH/\PY\to\VV}\xspace}
\newcommand{\hyvvq}{\ensuremath{\PH/\PY\to\VV\to\qqqq}\xspace}
\newcommand{\hvvq}{\ensuremath{\PH\to\VV\to\qqqq}\xspace}
\newcommand{\yvvq}{\ensuremath{\PY\to\VV\to\qqqq}\xspace}
\newcommand{\HHbbVV}{\ensuremath{{\PH\PH\to\bbbar\VV}}\xspace}
\newcommand{\HHbbVVq}{\ensuremath{{\PH\PH\to\bbbar(\VV\to\qqqq)}}\xspace}
\newcommand{\XHY}{\ensuremath{{\PX\to\PH\PY}}\xspace}
\newcommand{\XHYbbVV}{\ensuremath{{\PX\to(\PH\to\bbbar)(\PY\to\VV)}}\xspace}
\newcommand{\XHYbbVVq}{\ensuremath{{\PX\to(\PH\to\bbbar)(\PY\to\VV\to\qqqq)}}\xspace}

\newcommand{\TXbb}{\ensuremath{T_{\mathrm{Xbb}}}\xspace}
\newcommand{\TXbbbb}{\ensuremath{T_{\mathrm{Xbb}}^{\mathrm{bb}}}\xspace}
\newcommand{\THWW}{\ensuremath{T_{\mathrm{HVV}}}\xspace}
\newcommand{\THWWww}{\ensuremath{T_{\mathrm{HVV}}^{\mathrm{VV}}}\xspace}
\newcommand{\PXbb}{\ensuremath{P_{\mathrm{Xbb}}}\xspace}
\newcommand{\PXcc}{\ensuremath{P_{\mathrm{Xcc}}}\xspace}
\newcommand{\PXqq}{\ensuremath{P_{\mathrm{Xqq}}}\xspace}
\newcommand{\PQCD}{\ensuremath{P_{\mathrm{QCD}}}\xspace}
\newcommand{\PQCDcc}{\ensuremath{P_{\mathrm{QCDcc}}}\xspace}
\newcommand{\PQCDbb}{\ensuremath{P_{\mathrm{QCDbb}}}\xspace}
\newcommand{\PQCDc}{\ensuremath{P_{\mathrm{QCDc}}}\xspace}
\newcommand{\PQCDb}{\ensuremath{P_{\mathrm{QCDb}}}\xspace}
\newcommand{\PQCDothers}{\ensuremath{P_{\mathrm{QCDothers}}}\xspace}
\newcommand{\PTop}{\ensuremath{P_{\mathrm{Top}}}\xspace}
\newcommand{\PVJets}{\ensuremath{P_{\mathrm{V+Jets}}}\xspace}
\newcommand{\PggF}{\ensuremath{P_{\mathrm{ggF}}}\xspace}
\newcommand{\PVBF}{\ensuremath{P_{\mathrm{VBF}}}\xspace}
\newcommand{\PHWWqqqq}{\ensuremath{P_{\mathrm{HVV4q}}}\xspace}
\newcommand{\PHWWqqq}{\ensuremath{P_{\mathrm{HVV3q}}}\xspace}

\newcommand{\ggfbdt}{\ensuremath{\mathrm{BDT}_{\mathrm{ggF}}}\xspace}
\newcommand{\vbfbdt}{\ensuremath{\mathrm{BDT}_{\mathrm{VBF}}}\xspace}

\newcommand{\tbw}{\ensuremath{\PQt\to\PQb\PW}\xspace}
\newcommand{\tae}{\ensuremath{\PGt_{\Pe}}\xspace}
\newcommand{\tam}{\ensuremath{\PGt_{\PGm}}\xspace}
\newcommand{\tah}{\ensuremath{\PGt_{\mathrm{h}}}\xspace}

\newcommand{\mx}{\ensuremath{{m_X}}\xspace}
\newcommand{\mh}{\ensuremath{{m_H}}\xspace}
\newcommand{\my}{\ensuremath{{m_Y}}\xspace}
\newcommand{\mxmy}{\ensuremath{{\mx,\my}}\xspace}

\newcommand{\msd}{\ensuremath{{m_{\mathrm{SD}}}}\xspace}
\newcommand{\mreg}{\ensuremath{{m_{\mathrm{reg}}}}\xspace}
\newcommand{\mregbb}{\ensuremath{{m_{\mathrm{reg}}^{\mathrm{bb}}}}\xspace}
\newcommand{\mregvv}{\ensuremath{{m_{\mathrm{reg}}^{\mathrm{VV}}}}\xspace}
\newcommand{\mjj}{\ensuremath{{m^{\mathrm{jj}}}}\xspace}
\newcommand{\mjjvbf}{\ensuremath{{m^{\mathrm{jj}}_{\mathrm{VBF}}}}\xspace}
\newcommand{\detajjvbf}{\ensuremath{{|\Delta\eta^{\mathrm{jj}}_{VBF}|}}\xspace}

\newcommand{\pois}{\ensuremath{\mathrm{Pois}}\xspace}


\newcommand{\NA}{\ensuremath{\text{---}}\xspace}
\newcommand{\kt}{\ensuremath{k_{\mathrm{T}}}\xspace}
\newcommand{\eV}{\ensuremath{\,\text{e\hspace{-.08em}V}}\xspace}
\newcommand{\MeV}{\ensuremath{\,\text{Me\hspace{-.08em}V}}\xspace}
\newcommand{\GeV}{\ensuremath{\,\text{Ge\hspace{-.08em}V}}\xspace}
\newcommand{\TeV}{\ensuremath{\,\text{Te\hspace{-.08em}V}}\xspace}
\newcommand{\unit}[1]{\ensuremath{\text{\,#1}}\xspace}
\newcommand{\GEANTfour} {{\textsc{Geant4}}\xspace}
\newcommand{\PYTHIA} {{\textsc{pythia}}\xspace}
\newcommand{\HERWIG} {{\textsc{herwig}}\xspace}
\newcommand{\SHERPA} {{\textsc{sherpa}}\xspace}
\newcommand{\ttbar}{\ensuremath{\mathrm{t}\overline{\mathrm{t}}}\xspace}
\newcommand{\MADGRAPH} {{\textsc{MadGraph}}\xspace}
\newcommand{\MCATNLO} {{\textsc{MC@NLO}}\xspace}
\newcommand{\MGvATNLO}{\MADGRAPH{}5\_a\MCATNLO\xspace}
\providecommand{\mSD}{\ensuremath{m_{\mathrm{SD}}}\xspace}
\newcommand{\rpf}{\ensuremath{R_{\mathrm{P/F}}}\xspace}


\newcommand{\etarel}{\ensuremath{\eta^{\mathrm{rel}}}\xspace}
\newcommand{\phirel}{\ensuremath{\phi^{\mathrm{rel}}}\xspace}
\newcommand{\ptrel}{\ensuremath{p_{\mathrm{T}}^{\mathrm{rel}}}\xspace}

\newcommand{\real}[1]{\ensuremath{#1_{\mathrm{real}}}\xspace}
\newcommand{\gen}[1]{\ensuremath{#1_{\mathrm{gen}}}\xspace}
\newcommand{\fgdinf}{\ensuremath{\mathrm{FGD}_\infty}\xspace}

\newcommand{\wass}{\ensuremath{W_1}\xspace}
\newcommand{\wassm}{\ensuremath{W_1^{\mathrm M}}\xspace}
\newcommand{\wassp}{\ensuremath{W_1^{\mathrm P}}\xspace}
\newcommand{\wassefp}{\ensuremath{W_1^{\mathrm{EFP}}}\xspace}
\newcommand{\wassppt}{\ensuremath{W^{\ptrel}_{1p}}\xspace}

\newcommand{\jetnet}{\textsc{JetNet}\xspace}


\newcommand{\hhexp}{69\xspace}
\newcommand{\hhobs}{142\xspace}
\newcommand{\cvvexp}{0.9\xspace}
\newcommand{\cvvobs}{1.1\xspace}
\newcommand{\kvvexplims}{\ensuremath{[0.05, 1.98]}\xspace}
\newcommand{\kvvobslims}{\ensuremath{[-0.04, 2.05]}\xspace}

\newcommand{\ri}{\ensuremath{\rho_{\mathrm{in}}}\xspace}
\newcommand{\ro}{\ensuremath{\rho_{\mathrm{out}}}\xspace}
\newcommand{\CX}{\ensuremath{\mathcal{X}}\xspace}
\newcommand{\CY}{\ensuremath{\mathcal{Y}}\xspace}


\newcommand{\parenthesis}[1]{\left( #1 \right)}
\newcommand{\squarebracket}[1]{\left[ #1 \right]}
\newcommand{\xmark}{\text{\ding{55}}}
\newcommand{\order}[1]{\mathcal{O} \left( #1 \right)}


\makeatletter
\def\lasthline{\noalign{\hrule}\noalign{\vskip-2em\vskip-\arraycolsep}}
\def\fbottomrule{\noalign{\bottomrule}\noalign{\vskip-2em\vskip-\arraycolsep}}
\makeatother

\newcommand{\cbottomrule}{%
  \ifdefined\HCode  % slashed not supported by tex4ht
    \lasthline
  \else
    \bottomrule
  \fi
}

\maketitle

\begin{abstract}
\begin{doublespace}
% \hypersetup{urlcolor=awesome-nephritis}
\setlength{\parskip}{\baselineskip}
This is a series of walk-throughs of some concepts in statistics relevant for high energy physics (HEP). 
It is primarily based on (me trying to understand) Refs.~\cite{Cowan:2010js} and~\cite{Cranmer:2014lly}, and the CMS combine tool~\cite{CMS:2024onh}.
Both references are useful and highly recommended for reading; the emphasis here is heavily on \textbf{implementing and visualizing what they discuss in Python} --- as well as providing some heuristic derivations of some of their formulae --- as that is what personally helped me build an intuition for these concepts.

These notes are primarily intended to be followed interactively \href{https://rkansal47.github.io/stats-for-hep}{online} as Jupyter notebooks, but are also available as a \href{https://github.com/rkansal47/stats-for-hep/blob/gh-pages/stats-for-hep.pdf?raw=true}{PDF}.
Feedback and corrections are very welcome through e-mail or as \href{https://github.com/rkansal47/stats-for-hep/issues}{issues} on \href{https://github.com/rkansal47/stats-for-hep/}{Github}.
Finally, for those interested, some more notes and tutorials I have written up, such as on the standard model and machine learning in HEP, are available \href{https://rkansal47.github.io/}{here}. 
\end{doublespace}
\end{abstract}


{
\hypersetup{linkcolor=black}
\tableofcontents
}

\setlength{\parskip}{\baselineskip}

\begin{doublespace}
\input{stats.tex}
\end{doublespace}

\nocite{apsrev42Control}
\bibliography{bibliography}

\end{document}
